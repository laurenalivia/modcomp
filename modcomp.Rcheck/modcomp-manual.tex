\nonstopmode{}
\documentclass[a4paper]{book}
\usepackage[times,inconsolata,hyper]{Rd}
\usepackage{makeidx}
\makeatletter\@ifl@t@r\fmtversion{2018/04/01}{}{\usepackage[utf8]{inputenc}}\makeatother
% \usepackage{graphicx} % @USE GRAPHICX@
\makeindex{}
\begin{document}
\chapter*{}
\begin{center}
{\textbf{\huge Package `modcomp'}}
\par\bigskip{\large \today}
\end{center}
\ifthenelse{\boolean{Rd@use@hyper}}{\hypersetup{pdftitle = {modcomp: Linear Model Summary and Comparison Tables: A package to make regression model comparison even easier!}}}{}
\ifthenelse{\boolean{Rd@use@hyper}}{\hypersetup{pdfauthor = {Lauren Gould}}}{}
\begin{description}
\raggedright{}
\item[Title]\AsIs{Linear Model Summary and Comparison Tables: A package to make
regression model comparison even easier!}
\item[Version]\AsIs{0.0.0.9000}
\item[Description]\AsIs{User is able to extract model details while also allowing for a user-defined alpha value of significance. This offers a more flexible model summary output than the summary() function, where the alpha value is strictly set to 0.05. The output can be used to generate a more flexible model summary dataframe if used directly via 'extract_lm()', or this function will also be called as an internal function to both 'tablestack' and 'tablecomp', where the output summaries of desired model components can be compared directly between similar/nested models. This offers a great advantage when completing courses like 'Biostatistical Methods I or II', where it is common to have to compare and contrast various model components between various iterations of a model. These display options, via 'tablestack()' and/or 'tablecomp()' take away the inefficiency of scrolling between various model summaries, and instead offer side by side or a stacked view of output for easy and direct viewing/ comparison. }
\item[License]\AsIs{MIT + file LICENSE}
\item[Encoding]\AsIs{UTF-8}
\item[Roxygen]\AsIs{list(markdown = TRUE)}
\item[RoxygenNote]\AsIs{7.3.2}
\item[URL]\AsIs{}\url{https://github.com/laurenalivia/modcomp}\AsIs{}
\item[BugReports]\AsIs{}\url{https://github.com/laurenalivia/modcomp/issues}\AsIs{}
\item[Depends]\AsIs{knitr, R (>= 3.5)}
\item[Imports]\AsIs{stats}
\item[Suggests]\AsIs{rmarkdown}
\item[VignetteBuilder]\AsIs{knitr}
\item[LazyData]\AsIs{true}
\item[NeedsCompilation]\AsIs{no}
\item[Author]\AsIs{Lauren Gould [aut, cre]}
\item[Maintainer]\AsIs{Lauren Gould }\email{laurengould@ufl.edu}\AsIs{}
\end{description}
\Rdcontents{Contents}
\HeaderA{modcomp-package}{modcomp: Linear Model Summary and Comparison Tables: A package to make regression model comparison even easier!}{modcomp.Rdash.package}
\aliasA{modcomp}{modcomp-package}{modcomp}
\keyword{internal}{modcomp-package}
%
\begin{Description}
User is able to extract model details while also allowing for a user-defined alpha value of significance. This offers a more flexible model summary output than the summary() function, where the alpha value is strictly set to 0.05. The output can be used to generate a more flexible model summary dataframe if used directly via 'extract\_lm()', or this function will also be called as an internal function to both 'tablestack' and 'tablecomp', where the output summaries of desired model components can be compared directly between similar/nested models. This offers a great advantage when completing courses like 'Biostatistical Methods I or II', where it is common to have to compare and contrast various model components between various iterations of a model. These display options, via 'tablestack()' and/or 'tablecomp()' take away the inefficiency of scrolling between various model summaries, and instead offer side by side or a stacked view of output for easy and direct viewing/ comparison.
\end{Description}
%
\begin{Author}
\strong{Maintainer}: Lauren Gould \email{laurengould@ufl.edu}

\end{Author}
%
\begin{SeeAlso}
Useful links:
\begin{itemize}

\item{} \url{https://github.com/laurenalivia/modcomp}
\item{} Report bugs at \url{https://github.com/laurenalivia/modcomp/issues}

\end{itemize}


\end{SeeAlso}
\HeaderA{extract\_lm}{Extract Relevant lm Components}{extract.Rul.lm}
%
\begin{Description}
extract relevant summary components from a linear model
\end{Description}
%
\begin{Usage}
\begin{verbatim}
extract_lm(lm, alpha = 0.05, output = TRUE)
\end{verbatim}
\end{Usage}
%
\begin{Arguments}
\begin{ldescription}
\item[\code{lm}] linear model

\item[\code{alpha}] user-defined alpha; define the threshold for significance. Default is 0.05.

\item[\code{output}] controls whether the extracted components are published in the console, default is 'TRUE'.
\end{ldescription}
\end{Arguments}
%
\begin{Details}
The user is able to extract model details while also allowing for a user-defined alpha value of significance. This offers a more flexible model summary output
than the summary() function, where the alpha value is strictly set to 0.05. The output can be used to generate a more flexible model summary dataframe if used directly,
or this function will also be called as an internal function to both 'tablestack' and 'tablecomp', as it allows for more flexibility with the alpha value.
\end{Details}
%
\begin{Value}
dataframe of relevant lm components
\end{Value}
%
\begin{Examples}
\begin{ExampleCode}
#fit linear model
 data(faraway_teengamb)
 lmod<-lm(gamble~sex+status+income+verbal+sex:income, data=faraway_teengamb)
#extract components, supplying 'output=TRUE' to print output
 extract_lm(lmod)

\end{ExampleCode}
\end{Examples}
\HeaderA{faraway\_teengamb}{faraway\_teengamb}{faraway.Rul.teengamb}
%
\begin{Description}
Study of teenage gambling in Britain (included as 'teengamb' in the 'faraway' package).
\end{Description}
%
\begin{Format}
This data frame contains the following columns:
\begin{description}

\item[sex] 0=male, 1=female
\item[status] Socioeconomic status score based on parents' occupation
\item[income] in pounds per week
\item[verbal] verbal score in words out of 12 correctly defined
\item[gamble] expenditure on gambling in pounds per year

\end{description}

\end{Format}
%
\begin{Details}
The following documentation is what has been supplied in the 'faraway' package, version 1.0.8, where this dataset is directly sourced from. The teengamb data frame has 47 rows and 5 columns. A survey was conducted to study teenage gambling in Britain.
\end{Details}
%
\begin{Source}
Package 'faraway' version 1.0.8, Source listed: Ide-Smith \& Lea, 1988, Journal of Gambling Behavior, 4, 110-118
\end{Source}
%
\begin{Examples}
\begin{ExampleCode}
data(faraway_teengamb)
head(faraway_teengamb)
str(faraway_teengamb)
\end{ExampleCode}
\end{Examples}
\HeaderA{tablecomp}{Generate Model Comparison Table}{tablecomp}
%
\begin{Description}
Generate a single table for easy comparison between desired models, where each model occupies its own column/ set of columns.
\end{Description}
%
\begin{Usage}
\begin{verbatim}
tablecomp(
  ...,
  alpha_ = 0.05,
  modeltype = c("lm", "coxph"),
  comparison_value = c("coefs", "stderrs", "t_vals", "p_vals", "stars", "lower_confints",
    "higher_confints", "rsq", "adj.rsq", "aic", "alpha")
)
\end{verbatim}
\end{Usage}
%
\begin{Arguments}
\begin{ldescription}
\item[\code{...}] model(s) to display components for. can be just one, or as many as desired for side-by-side comparison of values

\item[\code{alpha\_}] user-defined alpha; define the threshold for significance. Default is 0.05.

\item[\code{modeltype}] 'lm' for linear model, 'coxph' for cox proportional hazards model.

\item[\code{comparison\_value}] model components to be compared in the output table. Can specify one, multiple, or all.
\end{ldescription}
\end{Arguments}
%
\begin{Details}
User can define which model components are of interest for a quick comparison, for example, model coefficient estimats ('coefs') and associated p values ('p\_vals'). All components extracted
using 'extract\_lm()' are available for comparison in this method. One quirk here is that models are displayed left to right in the order they are supplied in the funtion. A later version would fix the
lack of model labels, but for now this is the method to identify the models being compared--by order. It is also advised to supply the largest model first (especially if comparing nested models), as it will
ensure all predictor names will be displayed successfully. A later version would fix this quirk as well.
\end{Details}
%
\begin{Value}
table of relevant model components for a quick side-by-side comparison between models
\end{Value}
%
\begin{Examples}
\begin{ExampleCode}
#supply linear model(s) for output comparison
 data(faraway_teengamb)
 lmod1<-lm(gamble~sex+status+income+verbal+sex:status+sex:income+sex:verbal,data=faraway_teengamb)
 lmod2<-lm(gamble~sex+status+income+verbal+sex:income, data = faraway_teengamb)
 lmod3<-lm(gamble~sex+status+income+verbal, data = faraway_teengamb)
#determine what comparison_value(s) are important for the table, or user can do one comparison value
#per table to make viewing #even easier. Then create desired table(s) using 'comptable()'.
tablecomp(lmod1)
tablecomp(lmod1, lmod2, comparison_value= 'coefs')
tablecomp(lmod1, lmod2, lmod3, comparison_value= c('coefs', 'p_vals', 'stars'))

\end{ExampleCode}
\end{Examples}
\HeaderA{tablestack}{Generate a stack of model output tables for quick comparison.}{tablestack}
%
\begin{Description}
stack model information tables on top of each other to allow for a quick comparison between values.
\end{Description}
%
\begin{Usage}
\begin{verbatim}
tablestack(..., alpha_ = 0.05, modeltype = c("lm", "coxph"))
\end{verbatim}
\end{Usage}
%
\begin{Arguments}
\begin{ldescription}
\item[\code{...}] model(s) to display components for. can be just one, or as many as desired for a comparison of values

\item[\code{alpha\_}] user-defined alpha; define the threshold for significance. Default is 0.05.

\item[\code{modeltype}] 'lm' for linear model, 'coxph' for cox proportional hazards model.
\end{ldescription}
\end{Arguments}
%
\begin{Details}
function to generate the output of multiple models (of the same class) stacked on top of each other to make comparison of values easier. Or, only one model can be specified and the output
will still list the details for the singular model in an efficient way. This makes one large improvement over 'summary(lm)', as the user can define an alpha value beyond just a set 0.05.
\end{Details}
%
\begin{Value}
table(s) of relevant model components for a quick comparison. They are displayed in the order they are entered as input, with the top being the first model specified, and so on.
\end{Value}
%
\begin{Examples}
\begin{ExampleCode}
#fit linear models
data(faraway_teengamb)
lmod1<-lm(gamble~sex+status+income+verbal+sex:status+sex:income+sex:verbal,data=faraway_teengamb)
lmod2<-lm(gamble~sex+status+income+verbal+sex:income, data = faraway_teengamb)
#use 'tablestack()' to compare outputs displayed; can choose a user-defined alpha
#if default 0.05 is not the desired level.
 tablestack(lmod1, lmod2, alpha_= 0.1)

\end{ExampleCode}
\end{Examples}
\printindex{}
\end{document}
